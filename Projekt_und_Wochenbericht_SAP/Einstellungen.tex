% Vorlage für Praktikumsberichte
%
% Einstellungen.tex
%
% Hier werden allgemeine Einstellungen festgelegt.

\documentclass[a4paper,ngerman,13pt,bibliography=totoc,listof=totoc,numbers=noendperiod]{scrartcl}
%Startet das Dokument und regelt gleichzeitig Formatierungen wie Schriftgröße, Papierformat, Zahlenart, Sprache usw.

%Packages sind Erweiterungen des eigentlichen Programms und erleichtern Abläufe, Darstellungen, Formatierungen usw.
%Im Folgenden sind die Packages installiert, welche ich zum erstellen einer Bachelorarbeit einmal benötigt wurden.
%Es kann beliebig erweitert/reduziert werden.

%%%%%%%%%%%%%%%%%%%%%%%%%%%%%%%%%%%%%%%%%%%%%%%%%%%%%%%%%
%%%%%%%%%%%%%%%%%%%%%%%%%%%%%%%%%%%%%%%%%%%%%%%%%%%%%%%%%

\usepackage[ngerman]{babel}
% Paket für Deutsche Sprache (Übersetzungen von Chapter zu Kapitel, 
% richtige Umlaute, richtige % Silbentrennung)
% siehe auch http://de.wikipedia.org/wiki/Babel-System

\usepackage{titling}
%Hilft im Dokument auf den deklarierten Autor und Titel zuzugreifen 

\usepackage[style=numeric-comp,backend=biber,doi=false,isbn=false,maxnames=3,sorting=none]{biblatex}
\usepackage{csquotes}
\addbibresource{Referenzen.bib}
%Regelt die Erstellung des Literaturverzeichnisses sowie die Zitation im gesamten Bericht
%Greift gleichzeitig auf die Referenzen.bib genannte BibTex datei zu, in welcher die Quellen aufgeführt sind.

\usepackage{url}
%erzeugt schönere URLs (vorallem in den Quellen wichtig)

\usepackage{float}
%Hilft beim Anordnen von Bildern, Grafiken und Tabellen

\usepackage{abstract}
%ermöglicht das einfügen eines Abstract (Nicht relevant fürs Praktikum)

\usepackage{booktabs}
%ermöglicht das erstellen von schöneren Tabellen 

\usepackage{subfigure}
%lässt den Benutzer mehrere Grafiken/Bilder in einer Abbildung darstellen

\usepackage{setspace}
% Paket um den Zeilenabstand zu ändern
\onehalfspacing
% Zeilenabstand auf 1,5-fach setzen

\usepackage{pdfpages}
%ermöglicht das einbinden von eigenständigen PDF Dokumenten 

\usepackage{siunitx}
\sisetup{locale = DE}
%vereinfacht den benötigten Syntax zum Benutzen von SI Einheiten und Mathematischen Gleichungen mit dem deutschen Standart

\usepackage{tikz}
%Freies Anordnen von Bildern und Text in Relation zueinander, sowie Möglichkeiten Bilder zu bearbeiten
%Verfügt auserdem über die Möglichkeit Grafiken und Diagramme in Latex zu erstellen

\usepackage[format=plain,labelfont=bf]{caption}
\captionsetup[table]{position=above}
%verschönert die Abbildungs- und Tabellenbeschriftungen. Für Tabellen werden die Beschriftungen über der Tabelle angezeigt

\usepackage{icomma}
%Für den korrekten Abstand der Kommas in Dezimalzahlen und in Gleichungen

\usepackage{amsmath}
%elementares Paket für Gleichungen. Verschönert diese und erleichtert die Benutzung von Gleichungen.

\usepackage{graphicx}
%mehr Optionen beim Einbinden und anordnen von Bildern und Grafiken

\usepackage{setspace}
%Für den richtigen Zeilenabstand

%\usepackage[left=3.5cm, right=2.3cm, top=3cm, bottom=2.5cm]{geometry} %links mehr
\usepackage[left=3cm, right=3cm, top=3cm, bottom=2.5cm]{geometry} %links rechts gleich
%Regelt die Seitenabstände

\usepackage[headsepline=true, markcase=noupper]{scrlayer-scrpage}
%Regelt alle Einstellung bezüglich Kopf- und Fußzeile
\pagestyle{scrheadings}
\headheight 16pt
\setcounter{secnumdepth}{3}
% Setzt den Zähler für die Abschnittsnumerierung auf eine Tiefe von 10. Alle Abschnitte werden nummeriert.
\setcounter{tocdepth}{3}
% Setzt die Tiefe für das Inhaltsverzeichnis auf eine Tiefe von 10. Alle Abschnitte werden aufgeführt.
% Abschnittsnummer und Überschrift
\automark{section}
\ihead{\headmark}
\chead{}
\ohead{\pagemark}
\cfoot{}
\setlength{\parindent}{0pt}
%Einschub nach Absätzen und Zeilenumbrüchen
\renewcommand*{\headfont}{\normalfont}

\usepackage{enumitem}


%%%%%%%%%%%%%%%%%%%%%%%%%%%%%%%%%%%%%%%%%%%%%%%%%%%%%%%%%
%%%%%%%%%%%%%%%%%%%%%%%%%%%%%%%%%%%%%%%%%%%%%%%%%%%%%%%%%