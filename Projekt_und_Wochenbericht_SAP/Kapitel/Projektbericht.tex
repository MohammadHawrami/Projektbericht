% Vorlage für Praktikumsberichte
%
% Praktikum.tex
%
% Beschreibung der Praktikumstätigkeiten

\section{Projektbericht}

\begin{center}
    \large
    \textbf{Production ready cloud provisioning with Infrastructure as Code}
\end{center}

\subsection{Einleitung}
Punkte: \\
Motivation: Warum ist Cloud Provisioning in der modernen Softwarebereitstellung wichtig? \\
Ziel der Arbeit: Erweiterung des Crossplane Cloud Foundry Providers zur Unterstützung von MTAs
(Hier dann kurz Infrastructure as Data und Infrastructure as Code erwähnen) \\
Aufbau der Arbeit: Überblick über die Struktur und den Inhalt der Arbeit \\
Probleme: Warum CI/CD für Infrastruktur- und Applikations-Lifecycle nicht ausreicht\cite{lustigeCitation}

\subsection{Grundlagen und Technologien}

\subsubsection{CI/CD-Piepline: Rollen und Grenzen im Infrastrukturmanagement}

\subsubsection{Infrastructure as Code (IaC): z.B. Terraform, Pulumi}

\subsubsection{Infrastructure as Data (IaD): Kubernetes-CRDs als deklarativer Zustand}

\subsubsection{Kubernetes Operatoren und Controller Pattern: kurzer Überblick}

\subsubsection{Crossplane: Konzepte und Funktionen in diesem Projekt}

\subsubsection{SAP Cloud Orchestrator: Rolle bei der Ressourcenorchestrierung}

\subsubsection{Cloud Foundry: Konzept und Funktion in diesem Projekt (Primär auf MTA bezogen)}

\subsection{Konzept zur Erweiterung des bestehenden Cloud Foundry Providers durch die MTA Ressource}

\subsection{Implementierung der MTA Ressource:}
Designentscheidungen bei der Erweiterung des CF-Providers \\
CRD-Definition: YAML + Erläuterung \\
Reconcile-Loop / Controller handling des MTA-Lifecycle \\
Authentifizierung / Verbindung zu CF

\subsection{Fazit}

% Ende der Datei
