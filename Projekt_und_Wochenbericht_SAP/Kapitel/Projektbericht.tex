% Vorlage für Praktikumsberichte
%
% Praktikum.tex
%
% Beschreibung der Praktikumstätigkeiten

\section{Projektbericht}

\begin{center}
    \large
    \textbf{Production ready cloud provisioning with Infrastructure as Code}
\end{center}

\subsection{Einleitung}
Punkte: \\
Motivation: Warum ist Cloud Provisioning in der modernen Softwarebereitstellung wichtig? \\
Ziel der Arbeit: Erweiterung des Crossplane Cloud Foundry Providers zur Unterstützung von MTAs
(Hier dann kurz Infrastructure as Data und Infrastructure as Code erwähnen) \\
Aufbau der Arbeit: Überblick über die Struktur und den Inhalt der Arbeit \\
Probleme: Warum CI/CD für Infrastruktur- und Applikations-Lifecycle nicht ausreicht\cite{lustigeCitation}

\subsection{Grundlagen und Technologien}

\subsubsection{CI/CD-Piepline: Rollen und Grenzen im Infrastrukturmanagement}
Definition von CI/CD im Cloud-Kontext \\
Rolle bei Infrastrukturprovisionierung \\
Grenzen: Wo hört CI/CD auf, wo beginnt IaC?

\subsubsection{Infrastructure as Code (IaC): z.B. Terraform, Pulumi}
Was genau ist Infrastructure as Code \\
Prinzipien: deklarativ vs Imperativ \\
Tools: Terraform, Pulumi - Unterschiede und Use Cases \\
Vor- und Nachteile von IaC

\subsubsection{Infrastructure as Data (IaD): Kubernetes-CRDs als deklarativer Zustand}
Definition Infrastructur as Data: Abgrenzung zu IaC \\
CRDs(Custom Resource Definitions): Bedeutung und Funktion Kubernetes \\
Deklaritiver vs Imperativer Zustand: Vorteile des deklarativen Modells \\
Beispiele für CRD-Nutzung im Infrastrukturmanagement


\subsubsection{Kubernetes Operatoren und Controller Pattern: kurzer Überblick}

\subsubsection{Crossplane: Konzepte und Funktionen in diesem Projekt}
Was ist Crossplane: Kurze Einführung und Motivation \\
Crossplane-Architektur: Composition, Managed Resources, Providers \\
Einbindung von Providern (z.B. Cloud Foundry Provider) \\
Spezifische Relevanz für dieses Projekt: \\
    - Warum Crossplane für SAP / Cloud Foudry geeignet ist \\
    - Nutzung für MTA-Ressource

\subsubsection{SAP Cloud Orchestrator: Rolle bei der Ressourcenorchestrierung}
Kurzbeschreibung SAP Cloud Orchestrator: Funktion im SAP-Umfeld \\
Integration mit Kubernetes / Crossplane \\
Rolle im Projektkontext: Steuerung, Überwachung, Breitstellung

\subsubsection{Cloud Foundry: Konzept und Funktion in diesem Projekt (Primär auf MTA bezogen)}
Einführung Cloud Foundry (CF) \\
Was sind Multi-Target Applications (MTA): Aufbau, Vorteile, SAP-Kontex \\
Bereitstellung von MTAs in CF: Aktueller Stand, manuelle vs automatisierte Bereitstellung \\
Bezug zu Crossplane-Integration: Warum eine IaC-Lösung für MTA gebraucht wird

\subsection{Konzept zur Erweiterung des bestehenden Cloud Foundry Providers durch die MTA Ressource}
Ausgangslage:
    - Was bietet der CF Provider aktuell ohne MTA? \\
    - Beschreibung des bestehenden Cloud Foundry Providers \\
    - Funktionalitäten und Grenzen des bestehenden Providers \\
    - Warum ist eine Erweiterung notwendig? \\
Zielsetzung: Warum MTA als neue Ressoruce nötig ist \\
Architektur der Erweiterung: Wie wird die MTA-Ressource integriert? \\

\subsection{Implementierung der MTA Ressource:}
Designentscheidungen bei der Erweiterung des CF-Providers \\
CRD-Definition: YAML + Erläuterung \\
Reconcile-Loop / Controller handling des MTA-Lifecycle \\
Authentifizierung / Verbindung zu CF

\subsection{Fazit}

% Ende der Datei
